\chapter{Propagating the uncertainty of stereo images}
This chapter details the work conducted on the propagation of uncertainty from images into the cost curves of dense matching problems. We consider a simple model of uncertainty on the input images, a dependency model between the uncertain intensities of the images, and estimate the resulting uncertainty on the output cost curves. This chapter takes up work and data already published \cite{malinowski_copulas_2022, malinowski_uncertainty_2023, malinowski_robust_2024}.

\section{Sources of uncertainty in stereo matching}
\comroman{Maybe move this section after having presented the different ways to use Copulas and IP. 
Atmospheric correction, vibration, resolution of a pixel, discretization.
CO3D mission will not have the epipolar line correction problem that is encountered with pleiades images.
Epipolar line rectification for Pléiades is a problem that is not dealt here.}

To maintain simplicity in this section, we will not consider panchromatic images, such as Pléiades products, encoding the reflectance values as positive integer, usually contained in $[0, 5000]$. Instead, we consider grayscale images that have intensity levels quantified within the range $[0, 255]$, which will represent our measurable space $\X$. We hypothesize that a pixel's intensity value can deviate by no more than $1$ level from its observed value, with the observed value being the most likely. This hypothesis arises from the quantification of observed radiometric values into integers. To keep our explanation straightforward, we assume this simple hypothesis. Consequently, we model the uncertainty of each pixel $p\in I_L,I_R$ intensity with a possibility distribution $\pi$, centered around the observed intensity $i_p\in[0,255]$:
\begin{equation}
    \pi(i_p)=1,\quad \pi(i_p\pm1)=\alpha\,,
\end{equation}\label{eq:pixel_possibility}
with $\alpha \in [0,1]$. The $\pm$ indicates that both positive and negative values are considered. In our simulation, $\alpha = 0.3$ for pixels in the left image and $\alpha = 0.4$ for pixels in the right image. We use different values of $\alpha$ for the left and right images because the uncertainty model may vary between images due to differences in exposure, noise levels, or camera calibration. This model effectively states that we accept any probability distribution supported within $[i_p - 1, i_p + 1]$ where the probability measure $P$ satisfies $\{P(A) \leq \sup_{i \in A} \pi(i)\}$ as an acceptable model for our uncertainty. The mass distribution function $m_p$ associated to this credal set possesses two focal sets $a^p$:
\begin{eqnarray}
    &m_p(a^p_1=[\![i_p, i_p]\!])=1-\alpha\,\nonumber\\
    &m_p(a^p_2=[\![i_p-1, i_p + 1]\!])=\alpha\,\label{eq:pixel_mass}
\end{eqnarray}
with $[\![\cdot, \cdot]\!]$ referring to integer intervals. In particular, $[\![i_p, i_p]\!]$ correspond to the singleton $\{i_p\}$.

It is important to note that in this disparity estimation problem, we only account for the uncertainty in our input image intensities, without considering the uncertainty in our cost function's ability to correctly identify the true disparity as its minimum. In other words, we do not account for the uncertainty arising from the difference between ``two patches are very similar'' and ``the pixels at the center of the patches are homologous''. To better illustrate this, imagine a scenario where two pixels should be matched, but the surrounding patches are dissimilar. In this case, the cost function between those two patches would be high, potentially leading to the selection of a different patch with a lower cost function as the estimated disparity. The correct disparity would not be the minimum of the cost curve.

\section{Leveraging specificities to accelerate computations}
H Volume etc for SAD example, see IJAR.

\pagebreak