\chapter*{Résumé}\addcontentsline{toc}{chapter}{Résumé}
Actuellement, les Modèles Numériques de Surface (MNS) sont nécessaires pour de nombreuses applications, telles que la gestion des ressources en eau, le suivi de la biomasse, l'évaluation des dommages causés par les catastrophes naturelles ou la planification urbaine. Les MNS peuvent principalement être produits par interférométrie \acrshort{radar}, photogrammétrie ou en utilisant des instruments \acrshort{lidar}. Dans ce contexte, le \acrshort{cnes} et Airbus préparent le lancement de la constellation de satellites \acrshort{co3d} afin d'assurer la production massive de MNS à haute résolution par photogrammétrie. Fournie avec le MNS, une carte de performance permettra de caractériser les erreurs liées aux incertitudes dans les données d'entrée ainsi qu'aux incertitudes des méthodes utilisées.

L'objectif de cette thèse et de caractériser l'incertitude associée à la production de MNS par photogrammétrie. Nous utilisons des modèles d'incertitude spécifiques, à savoir des probabilités imprécises, et plus particulièrement des distributions de possibilité, afin de caractériser l'incertitude résultant du traitement des images stéréo. Ces modèles définissent des ``ensembles crédaux'', qui sont des ensembles convexes de distributions de probabilité. L'intérêt de ces ensembles crédaux est d'être mieux adaptés pour représenter l'incertitude résultant de connaissances incomplètes ou imparfaites, par rapport aux simples distributions de probabilité. En présence de plusieurs sources d'incertitudes, il est également nécessaire de considérer leurs relations de dépendance. Pour cela, il est courant d'utiliser des copules, qui sont des modèles représentant la dépendance entre plusieurs variables aléatoires. Dans cette thèse, trois méthodes distinctes sont introduites afin de joindre des ensembles crédaux marginaux en des ensembles crédaux multivariés à l'aide de copules. Les relations entre ces méthodes sont ensuite étudiées pour des copules spécifiques ainsi que pour différents modèles de probabilités imprécises. Une application de ces ensembles crédaux multivariés est ensuite proposée, afin de propager l'incertitude d'images stéréo dans un problème d'appariement. Différentes optimisations et façons de faciliter la propagation de l'incertitude sont présentées. La propagation correcte de l'incertitude est enfin validée à l'aide de méthodes de Monte-Carlo.

Une seconde contribution de cette thèse concerne la modélisation de l'incertitude intrinsèque de l'algorithme d'appariement en utilisant des distributions de possibilité. Une méthode est proposée pour générer des intervalles de confiance associés aux résultats de l'étape d'appariement, et ces intervalles sont propagés jusqu'à la fin du pipeline stéréo, produisant ainsi des intervalles de confiance d'élévation pour les MNS. La taille et la précision de ces intervalles est évaluée en utilisant des images satellites réelles et des MNS pour lesquels une vérité terrain est disponible. Les intervalles ainsi créés contiennent correctement la vérité terrain au moins 90 $\%$ du temps.
\clearpage

\chapter*{Abstract}\addcontentsline{toc}{chapter}{Abstract}
Currently, Digital Surface Models (\acrshort{dsm}s) are required in many applications, such as for managing water resources, monitoring biomass, evaluating damages caused by natural catastrophes, or for urban planning. \acrshort{dsm}s can mainly be produced by \acrshort{radar} interferometry, photogrammetry or \acrshort{lidar} scanning. In this context, \acrshort{cnes} and Airbus are planning the launch of the \acrshort{co3d} constellation of satellites to massively provide highly accurate \acrshort{dsm}s using photogrammetry. A performance map will also be provided alongside the \acrshort{dsm} to characterize potential errors resulting from the uncertainty on input data or on its processing.

The objective of this thesis is to characterize the uncertainty associated with the production of \acrshort{dsm}s using photogrammetry. To do so, special uncertainty models, namely imprecise probabilities, and more specifically possibility distributions, are employed to characterize the uncertainty arising from stereo images processing. Those models define credal sets, which are convex sets of probability distributions. Credal sets are well-suited to represent uncertainty resulting from incomplete or imperfect knowledge, which can be a limitation for a single probability distribution. In the presence of multiple sources of uncertainty, their dependency must also be considered. For this purpose, it is possible to consider copulas, which are models used to represent the dependency between multiple random variables. In this thesis, three different methods are introduced to join marginal credal sets into multivariate credal sets using copulas. The relationships between those methods are then investigated, for specific copulas and different models of imprecise probabilities. An application of those multivariate credal sets is then proposed, for propagating the uncertainty of stereo images in a dense stereo-matching problem. Different optimizations and ways to facilitate the uncertainty propagation are presented. The correct uncertainty propagation is validated using Monte Carlo sampling.

A second contribution of this thesis concerns the uncertainty modeling of the dense-matching algorithm itself using possibility distributions. A method is presented for generating confidence intervals associated with the results of the dense-matching step. Those intervals are then propagated to the end of the stereo pipeline, therefore producing elevation confidence intervals for the \acrshort{dsm}s. The size and accuracy of intervals are then evaluated, using real satellites images and \acrshort{dsm}s for which a ground truth is available. Elevation intervals correctly contain the ground truth at least $90\%$ of the time.

\clearpage

\chapter*{Foreword}\addcontentsline{toc}{chapter}{Foreword}

Before delving into the subject of this manuscript, we would like to give some advice on how to efficiently navigate through it. When writing this thesis, we made extensive use of the \textit{hyperref} package, so that reading it on a PDF viewer was made easier. You can thus click on citations, figure numbers, equation numbers, chapters and sections numbers, acronyms \etc to directly jump to the concerned part. When following a reference to a citation, a previous chapter, equations or figures located in a different part of the manuscript, it can be a arduous process to go back to the section you were reading. Depending on the OS of your computer and the app used to read the PDF document, there usually exist shortcuts to jump back to the previous view. This allows to quickly switch back and forth between chapters and sections.

For instance, imagine that you are in \Cref{chap:epistemic_uncertainty} and we make a reference to an equation from \Cref{chap:representation_of_uncertainty}. If you do not recall the equation, and quickly want to see what it is about, simply click on the hyperlink to directly go to the relevant equation from \Cref{chap:representation_of_uncertainty}. Then use your system's shortcut to go back to where you were in \Cref{chap:epistemic_uncertainty}. 
\begin{itemize}
    \item Using Acrobat Reader: the shortcut \keys{\Altwin + \arrowkeyleft} (left arrow key) on Windows or Linux brings you to the previous view after clicking on a hyperlink. Afterwards, you can alternate views with \keys{\Altwin + \arrowkeyright} and \keys{\Altwin + \arrowkeyleft}. On macOS, the \keys{\Altwin} key is replaced by the \keys{\cmd} key.
    \item Using Preview on macOS, you can add the \menu{Page History} button to the toolbar, by right-clicking on the toolbar and selecting \menu{Customize Toolbar}
    \item Using Okular on Linux, \keys{\Altwin + \shift + \arrowkeyleft} (left arrow key) brings you to the previous view after clicking on a hyperlink. Afterward, you can alternate views with \keys{\Altwin + \shift + \arrowkeyright} and \keys{\Altwin + \shift + \arrowkeyleft}
\end{itemize}
Hopefully, this makes the reading of this thesis a more pleasant experience.

\clearpage

\renewcommand{\epigraphsize}{\normalsize}
\setlength{\epigraphrule}{1pt}
\setlength{\epigraphwidth}{0.48\linewidth}
\vspace*{\fill}
\epigraph{You never talk of likelihoods on Arrakis.\\You speak only of possibilities.}{Frank Herbert, \textit{Dune}}
\vspace*{\fill}

\clearpage